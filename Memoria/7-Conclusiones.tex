\chapter{Conclusiones}\label{cap.conclusiones}

Una vez se ha detallado en los últimos capítulos todo el software desarrollado, ha llegado el momento de analizar las contribuciones realizadas y verificar si se han cumplido los objetivos marcados.

\section{Contribuciones}
El principal objetivo marcado en este trabajo era enriquecer las herramientas existentes en la plataforma JdeRobot con nuevas herramientas con tecnologías web. Se puede concluir que se ha alcanzado el objetivo satisfactoriamente realizando la conversión de los visores para su utilización con Electron y ROS, se ha aportado un nuevo visor de elementos 3D y se ha elaborado un nuevo driver para complementar los drivers para fuentes de video ya existentes en la plataforma JdeRobot.

El primer subobjetivo era modificar los tres visores web existentes previamente en la plataforma JdeRobot para que pudieran ser usados tanto en el navegador como con Electron y, a su vez, que pudieran conectarse mediante los middleware ICE y ROS. Como se ha explicado en el capitulo 4 de este trabajo, esta meta se ha alcanzado de manera satisfactoria, teniendo ahora nuevas versiones de los visores llamadas \texttt{CamVizWeb}, \texttt{TurtlebotVizWeb} y \texttt{DroneVizWeb}. Estos tres visores son capaces de comunicarse mediante los middleware de comunicación ICE y ROS, dotándoles de una gran versatilidad al poder conectar un amplio abanico de drivers robóticos.

La segunda meta era crear un nuevo visor 3D con WebGL que sustituyera al visor existente desarrollado en C++, utilizado en la práctica de Robotics Academy de reconstrucción 3D. Este subobjetivo se ha alcanzado con la herramienta web \texttt{3DVizWeb} y, como se ha visto en el capitulo 5, no solo hasta el punto que se había marcado inicialmente (visualización de puntos), sino que se decidió ampliar el visor para que pudiera mostrar también segmentos y objetos 3D. No ha sido posible imitar completamente el funcionamiento del visor anterior, en el nuevo cambia la manera de comunicarse con las aplicaciones. El visor 3D es el encargado de llevar la iniciativa y debe ser él que solicite los elementos al servidor, y no que los reciba sin más como funcionaba anteriormente. Esta herramienta es muy útil para que otros desarrolladores e investigadores puedan representar escenas adquiridas mediante sensores en un mundo tridimensional.

La tercera meta fijada era crear un componente que pudiera obtener y enviar las imágenes obtenidas de una fuente de video WebRTC, complementando a los componentes \texttt{cameraserve}r y \texttt{cameraserver\_py} de JdeRobot que realizan la misma función pero desarrollados con C++ y Python respectivamente. Este subobjetivo se ha alcanzado con el componente \texttt{CamServerWeb} que, como se ha mostrado en el capitulo 6, obtiene imágenes mediante WebRTC de los dispositivos de video conectados y los transmite mediante el middleware de comunicación ROS. Con este componente se da la posibilidad de que cualquier aplicación de visión artificial pueda recibir las imágenes grabadas por las cámaras conectadas a un robot.

Además se ha conseguido que todas las herramientas puedan ser ejecutadas con Electron, lo que permite utilizarlas como una aplicación de escritorio, facilitando su uso por otros usuarios. Gracias a ello, todas las herramientas pueden ejecutarse a través de dos vías (Electron y navegador web) y son multiplataforma.

Analizando todo lo realizado y tras probar su eficiencia mediante las experimentaciones que han resultado exitosas en todos los casos, se puede concluir que se ha enriquecido con tecnologías web la plataforma JdeRobot, que hasta ahora su desarrollo principal era con C++ y Python.

Finalmente, a título personal, he ampliado mis capacidades con tecnologías web y aprendido el uso del entorno Electron que es muy útil para elaborar aplicaciones de escritorio pero programadas como si fueran web. He conocido el mundo de la robótica y como mediante middlewares como ICE o ROS es posible conectar sensores o actuadores a una aplicación para compartir información entre ambos.

\section{Trabajos Futuros}
Todas las herramientas creadas son completamente funcionales y cualquier usuario puede utilizarlas sin problemas, sin embargo existen varios aspectos donde se pueden extender para hacerlas aún más útiles.

\begin{itemize}
\item La primera posibilidad es permitir que tanto la herramienta \texttt{CamVizWeb} como el componente \texttt{CamServerWeb} sean compatibles con las imágenes en crudo de ROS y no solo imágenes comprimidas como ahora. Esta mejora permitiría conectar este visor al resto de drivers de fuentes de video.
\item Otra posibilidad a modificar es conseguir que \texttt{3DVizWeb} reciba los elementos sin que haya realizado la petición previamente, de modo que la iniciativa la lleve enteramente la aplicación y no el visor. Este hecho permitirá que pueda conectarse al visor más de una aplicación simultáneamente, además reducirá el retardo del visor al reducir a la mitad el intercambio de mensajes.
\item Empaquetar las herramientas web mediante paquetes npm, de modo que permita a otros usuarios descargar e instalar fácil y rápidamente las herramientas para ser usadas.
\end{itemize}


