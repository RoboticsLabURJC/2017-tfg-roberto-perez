\chapter*{Resumen}
\setlength{\parskip}{1ex}
La plataforma JdeRobot nació como un paquete de softwares para el desarrollo de aplicaciones robóticas y visión artificial. El propósito de la plataforma es crear herramientas y controladores que faciliten la conexión con componentes hardware. La plataforma esta principalmente desarrollada en C++ y Python.

El propósito de este proyecto es facilitar el uso de las diferentes aplicaciones mediante el uso de tecnologías web en el front-end. Esta tecnología nos permite la ejecución de las diferentes herramientas con independencia del sistema operativo en que se está ejecutando. Además, utilizando el framework Electron, seremos capaces de crear aplicaciones de escritorio evitando de este modo problemas de compatibilidades entre los diferentes navegadores web que hay actualmente.

El desarrollo de este proyecto se realizará utilizando JavaScript, HTML5 y CSV en el cliente, y NodeJS en el servidor, sobre el cual está implementada la tecnología de Electron. Además me apoyare en los middleware ZeroC ICE y ROS para la interconexión cliente-servidor. 

La primera parte del proyecto está centrada en la adaptación de los Visores existentes en la plataforma JdeRobot para que funcionen como aplicaciones de escritorio. Estos visores son Turtlebotviz, Droneviz y  Camviz. Para finalizar esta parte, me centrare en este último visor para hacerlo funcionar con los dos principales middleware de comunicación: ZeroC ICE (como hasta ahora) y ROS.

La segunda parte, consiste en la creación de un nuevo visor web que mostrara objetos 3D, desde un simple punto o segmento, hasta objetos más complejos que se obtienen a partir de modelos 3D. Una de las funciones de este visor es la de ser utilizado en la práctica de JdeRobot Academy “Reconstrucción 3D”.

El último propósito de este proyecto es la elaboración de un nuevo componente para la plataforma JdeRobot. Este nuevo componente será un servidor de imágenes obtenidas a través de cámaras web y servidas a los diferentes visores. Este componente utiliza como middleware de comunicación ROS.
